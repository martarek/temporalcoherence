\documentclass{article} % For LaTeX2e
\usepackage{nips13submit_e,times}
\usepackage{hyperref}
\usepackage{url}
\usepackage{booktabs}
\usepackage{graphicx}
\usepackage{amsfonts}
%\documentstyle[nips13submit_09,times,art10]{article} % For LaTeX 2.09


\title{Projet}


\author{
Olivier Gagnon \And Bernard Lebel
}

% The \author macro works with any number of authors. There are two commands
% used to separate the names and addresses of multiple authors: \And and \AND.
%
% Using \And between authors leaves it to \LaTeX{} to determine where to break
% the lines. Using \AND forces a linebreak at that point. So, if \LaTeX{}
% puts 3 of 4 authors names on the first line, and the last on the second
% line, try using \AND instead of \And before the third author name.

\newcommand{\fix}{\marginpar{FIX}}
\newcommand{\new}{\marginpar{NEW}}

\nipsfinalcopy % Uncomment for camera-ready version

\begin{document}


\maketitle

\begin{abstract}

\end{abstract}

\section{Introduction}

% Describe the problem or application that your project is concerned with.
Object recognition is a classification problem in which we try to identify an object in a picture using classification. For this purpoise, several database can be used to help the creation of algorithms to solve this problem. However, these database are very expensive. To alleviate this problem, algorithms can use non-labeled data to extract information useful to this problem.

One of the information that can prove useful is temporal coherence. For example, in a video sequence, it is likely that two adjacent frames contains the same objects rather than radically different ones. This is true to many other domains were data is sequential in nature.

The method chosen to solve this problem is to use temporal coherence as a regularizer to exploit this information in unlabeled temporal data\cite{?}.
%Mention the method you have chosen and why it solves your problem or is a good choice for your application.

\section{Description}
%-Describe in detail your method / algorithm implementation. Here are some things that you could discuss :
%    – Description of the data.
%    – Description and notation for the inputs.
%    – Description and notation for the targets.
The dataset used in this project is the COIL100 dataset\cite{?}. This dataset presents the pictures of 100 objects which are placed on a turntable and rotated by increments of 5 degrees by picture. It contains 72 images by object, for a total of 7200 pictures. The input consists of rgb images in the gif format and the target consists of an integer from 1 to 100.

%– Write a description of the general principles behind your approach. Here are some things that you could discuss :
%    – Type of learning (supervised / unsupervised, discriminative / generative).
%    – Intuition behind the training objective your algorithm optimizes.
%    – Intuition behind the architecture of the neural network.
The algorithm implemented in this project is a semi-supervised algorithm. The base algorithm used is a convolutionnal-neural-net trained as a classifier on the images. However, a regularizer is then applied to ensure that the convolutionnal-neural-net doesn't act in a radically different manner for two consecutives unlabelled video-frame. A second regularization is also applied to ensure that two non-consecutives frame do not act in the same way.
%– Provide a detailed description of your algorithm which implements these principles. Your description should allow a person to reimplement your method from your description. Here are some things that you could discuss :
%    – Objective optimized during training.
%    – Optimization technique used.
%    – Description of gradients.
%    – Architecture of the neural network.
%    – Training procedure (one phase of training or training in several phases).
%    – Description of the use of the network to make a prediction on new data.
%    - Description of hyper-parameters.
%    – Pseudocodes of your algorithm.
\section{Results}
%– Describe your experimental protocol (proportions used to divide the data into training, validation and test sets, data preprocessing, etc.).
%– Describe your procedure for selecting hyper-parameters.
%– Present and analyze the results for different hyper-parameters on the validation and training sets.
%– Present any results that validate the correctness of your implementation.
%– Present the results of your algorithm on the test set and make a comparison with an alternative, simple baseline.
\end{document}
